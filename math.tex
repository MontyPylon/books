\documentclass{article}
\usepackage[utf8]{inputenc}

\usepackage{amsmath}
\usepackage{amssymb}
\usepackage{float}
\usepackage{setspace}
\usepackage{fixltx2e}
\usepackage{graphicx}
\usepackage{multicol}
\usepackage[normalem]{ulem}
\usepackage{color}
\usepackage{hyperref}
\usepackage{caption}
\usepackage{comment}
\usepackage{multirow}
\usepackage[makeroom]{cancel}
\usepackage{braket}
\usepackage{amsthm}
\usepackage{mathtools}
\DeclarePairedDelimiter\ceil{\lceil}{\rceil}
\DeclarePairedDelimiter\floor{\lfloor}{\rfloor}

\newtheorem{theorem}{Theorem}[section]
\numberwithin{theorem}{subsection}
\numberwithin{theorem}{subsubsection}

\newtheorem{lemma}{Lemma}[section]
\numberwithin{lemma}{subsection}
\numberwithin{lemma}{subsubsection}

\theoremstyle{definition}
\newtheorem{definition}{Definition}[section]
\numberwithin{definition}{subsection}
\numberwithin{definition}{subsubsection}

\title{The Math Book}
\author{Jeremy Cook}
\date{December 2017}

\begin{document}

\maketitle

\tableofcontents
\newpage

\section{Introduction}
This book is intended to store all the useful and beautiful mathematical tools learned throughout my education which I needlessly forget from time to time.
\section{General Math}
\subsection{Long Division}
\section{Calculus}
\subsection{Integration Table}

\subsection{Epsilon Delta Proof of Limit} 
The proof that a function has a limit using the epsilon delta proof:
\begin{gather*}
    \lim_{x \rightarrow c} f(x) = L
\end{gather*}

\noindent means that

\begin{gather*}
    \forall \epsilon > 0,\ \exists \delta > 0,\ s.t.\ \forall x,\ 0 < |x - c| < \delta \implies |f(x) - L| < \epsilon.
\end{gather*}

\subsection{Integration by Parts}
The product rule says that 

\begin{gather*}
    \frac{d}{dx}(fg) = \frac{df}{dx}g + f\frac{dg}{dx}
\end{gather*}

\noindent from which it follows that 

\begin{gather*}
    \int_{a}^{b} f \frac{dg}{dx}dx = fg|_{a}^{b} - \int_{a}^{b} \frac{df}{dx}gdx
\end{gather*}

\noindent which can be written more simply as

\begin{gather*}
    \int udv = uv - \int vdu.
\end{gather*}

If the boundary conditions of $uv$ are known, then integration by parts can be very useful, or if you would like to switch which variable is a differential.

\subsection{Differentiation Under the Integral Sign}
Also called the Leibniz integral rule, differentiation under the integral sign states that an integral of the form

\begin{gather*}
    \int_{a(x)}^{b(x)} f(x,t)\,dt,
\end{gather*}

\noindent where $-\infty < a(x),\ b(x) < \infty$, the derivative of this integral is expressible as

\begin{gather*}
    \frac{d}{dx} \left (\int_{a(x)}^{b(x)} f(x,t)\,dt \right ) = \\ f(x,b(x))\cdot \frac{d}{dx} b(x) - f(x,a(x))\cdot \frac{d}{dx} a(x) + \int_{a(x)}^{b(x)}\frac{\partial}{\partial x} f(x,t) \,dt,
\end{gather*}

Notice that if $a(x)$ and $b(x)$ are constants rather than functions of $x$, we have a special case of Leibniz's rule

\begin{gather*}
    \frac{d}{dx} \left(\int_{a}^{b} f(x,t)\,dt \right)= \int_{a}^{b}\frac{\partial}{\partial x} f(x,t) \,dt.
\end{gather*}

\subsection{Series and Sums}
\begin{itemize}
    \item \textbf{Geometric Sum/Series} - A geometric series is a series with a constant ratio between successive terms. For example,
    \begin{equation}
        \frac{1}{2} + \frac{1}{4} + \frac{1}{8} + \frac{1}{16} + ...
    \end{equation}
    The common ratio $r$ is the ratio between an two successive terms. If $r$ is within the range $(-1,1)$, then the terms approach zero and will converge. Outside this range, the sum will converge. The start term $a$ can be used to write the sum as:
    
    \begin{equation}
        a + ar + ar^2 + ar^3 + ... = \sum_{n=0}^{\infty} ar^n
    \end{equation}
    
    If the sum converges, then the sum is equal to:
    
    \begin{equation}
        \sum_{n=0}^{\infty} ar^n = \frac{a}{1 - r},\ \text{for}\ |r| < 1
    \end{equation}
\end{itemize}

\subsection{Exact Differential}
\indent Consider a function $F(x_{1},x_{2})$ that depends on two independent variables $x_{1}$ and $x_{2}$. The differential of $F$ is defined as

\begin{gather*}
    dF = \left ( \frac{\partial F}{\partial x_{1}} \right )_{x_{2}} dx_{1} + \left ( \frac{\partial F}{\partial x_{2}} \right )_{x_{1}} dx_{2} 
\end{gather*}

\noindent where the subscript on the partial derivatives indicates holding that variable fixed(this is the normal definition of a partial derivative, but it is good to be explicit in thermodynamics). If $F$ and its derivatives are continuous and 

\begin{gather*}
    \left [ \frac{\partial}{\partial x_{1}}\left ( \frac{\partial F}{\partial x_{2}} \right )_{x_{1}}\right ]_{x_{2}} = \left [ \frac{\partial}{\partial x_{2}}\left ( \frac{\partial F}{\partial x_{1}} \right )_{x_{2}}\right ]_{x_{1}}
\end{gather*}

\noindent then $F$ is an exact differential. The fact that $dF$ is exact has the following consequences:

\begin{itemize}
    \item[(i)] The value of the integral $F(b) - F(a) = \int_{a}^{b} dF$ is independent of the path taken between $a$ and $b$, and depends only on the endpoints $a$ and $b$.
    \item[(ii)] The integral of $dF$ around a closed path $C$ is zero: $\int_{C} dF = 0.$
    \item[(iii)] If one knows only the quantity $dF$, then the function $F$ can be found to within an additive constant.
\end{itemize}

\indent If $F$ depends on more than two variables, then the statements about can be generalized in a simple way. Let $F = F(x_{1},x_{2},...,x_{n})$, then the differential of $F$ can be written as

\begin{gather*}
    dF = \sum_{i = 1}^{n} \left ( \frac{\partial F}{\partial x_{i}} \right )_{\{x_{j \neq i}\}}dx_{i}.
\end{gather*}

The differential $dF$ is exact if for any pair of variables the following property holds:

\begin{gather*}
    \left [ \frac{\partial}{\partial x_{i}}\left ( \frac{\partial F}{\partial x_{j}} \right )_{\{x_{k \neq j}\}}\right ]_{\{x_{k \neq i}\}} = \left [ \frac{\partial}{\partial x_{j}}\left ( \frac{\partial F}{\partial x_{i}} \right )_{\{x_{k \neq i}\}}\right ]_{\{x_{k \neq j}\}}
\end{gather*}

\begin{gather}
    \left(\frac{\partial x}{\partial y}\right)_{z}\left(\frac{\partial y}{\partial z}\right)_{x}\left(\frac{\partial z}{\partial x}\right)_{y} = -1
\end{gather}

\begin{comment}
======================================================================================================================================================================================================================================================================================DIFFERENTIAL EQUATIONS=================================================== ====================================================================================================================================================================================================================================
\end{comment}

\section{Differential Equations}

TODO 
\subsection{Separation of Variables}

\begin{comment}
=================================================================================================================================================================================================================================================================== LINEAR ALGEBRA ================================================= ====================================================================================================================================================================================================================================
\end{comment}

\section{Linear Algebra}
\subsection{Linear Form}
A linear functional or linear form is a linear map from a vector space to a field of scalars. In general, if $V$ is a vector space over a field $k$, then a linear functional $f$ is a function from $V$ to $k$ that is linear:
\begin{gather}
    f(\vec{v}+\vec{w}) = f(\vec{v})+f(\vec{w})\quad \forall \vec{v}, \vec{w}\in V\\
    f(a\vec{v}) = af(\vec{v})\quad \forall \vec{v}\in V, a\in k.
\end{gather}
The set of all linear functionals from $V$ to $k$, Hom$_{k}(V,k)$, forms a vector space over k with the addition of the operations of addition and scalar multiplication (defined pointwise).  This space is called the dual space of V, or sometimes the algebraic dual space, to distinguish it from the continuous dual space.


\begin{comment}
======================================================================================================================================================================================================================================================================================VECTOR CALCULUS=================================================== ====================================================================================================================================================================================================================================
\end{comment}

\section{Vector Calculus}

\subsection{Partial Derivative}
\noindent \textbf{Total derivative vs Partial derivative}\\
A partial derivative  operates under the assumption that you hold all variables fixed while one changes. When computing a total derivative, you allow changes in one variable to affect any of the others.

\begin{comment}
======================================================================================================================================================================================================================================================================================DISCRETE MATH=================================================== ====================================================================================================================================================================================================================================
\end{comment}

\section{Discrete Math}

\subsection{Propositional Logic}

\begin{flushleft}
\textbf{Propositional}
\end{flushleft}
A statement that is either true or false. Example: Austin in the capital of Texas.


\begin{flushleft}
\textbf{Logical Connective}
\end{flushleft}
A logical operator that takes one or two propositional variables and applies an operation on them to produce a single true or false value. The connectives are:


\begin{table}[H]
  \begin{center}
    \begin{tabular}{l|c|l} % <-- Alignments: 1st column left, 2nd middle and 3rd right, with vertical lines in between
      \textbf{Connective} & \textbf{Symbol} & \textbf{English}\\
      \hline
      Negation & $\neg$ & not\\
      Conjunction & $\wedge$ & and\\
      Disjunction & $\vee$ & or\\
      Implication & $\implies$ & implies\\
      Biconditional & $\iff$ & iff\\
      Exclusive Or & $\oplus$ & XOR\\
      Equality & $\equiv$ & equivalent\\
      Not And & $\uparrow$ & NAND\\
    \end{tabular}
  \end{center}
\end{table}

\begin{flushleft}
\textbf{Truth Tables}
\end{flushleft}

\begin{table}[H]
    \begin{minipage}{.25\linewidth}
      \caption*{\textbf{AND}}
      \centering
        \begin{tabular}{c|c|c}
             p & q & p $\vee$ q\\
             \hline
             T & T & T \\
             T & F & F \\
             F & T & F \\
             F & F & F \\
        \end{tabular}
    \end{minipage}%
    \begin{minipage}{.25\linewidth}
      \centering
        \caption*{\textbf{OR}}
        \begin{tabular}{c|c|c}
             p & q & p $\vee$ q\\
             \hline
             T & T & T \\
             T & F & T \\
             F & T & T \\
             F & F & F \\
        \end{tabular}
    \end{minipage}%
    \begin{minipage}{.25\linewidth}
      \centering
        \caption*{\textbf{XOR}}
        \begin{tabular}{c|c|c}
             p & q & p $\vee$ q\\
             \hline
             T & T & F \\
             T & F & T \\
             F & T & T \\
             F & F & F \\
        \end{tabular}
    \end{minipage}%
    \begin{minipage}{.25\linewidth}
      \centering
        \caption*{\textbf{NAND}}
        \begin{tabular}{c|c|c}
             p & q & p $\vee$ q\\
             \hline
             T & T & F \\
             T & F & T \\
             F & T & T \\
             F & F & T \\
        \end{tabular}
    \end{minipage}%
\end{table}


\section{Set Theory}
A set is an unordered collection of distinct objects. Some important sets include:

\begin{table}[H]
    \centering
    \begin{tabular}{l l l}
         $\emptyset : $ empty set & $\mathbb{Z}:$ integers & $\mathbb{Z}^{+}:$ positive integers \\
         $\mathbb{Z}^{-}:$ negative integers & $\mathbb{N}:$ natural numbers & $\mathbb{R}:$ real numbers \\
         $\mathbb{Q}:$ rational numbers & $\mathbb{C}:$ complex numbers & $\Omega:$ universal set
    \end{tabular}
    \label{tab:my_label}
\end{table}

\noindent{\textbf{Cardinality}} \\
\indent $|S| = $ number of elements in the set \\

\noindent{\textbf{Universal Set}} \\
\indent Equivalent to universe of discourse. All possible elements are in this set.\\

\noindent{\textbf{Singleton Set}} \\
\indent Set with cardinality of 1, meaning a single element in the set.\\

\noindent \textbf{Subset} \\
\indent A set A is a subset of a set B, or equivalently B is a superset of A, if A is ``contained" inside B, that is, all elements of A are also elements of B. A and B may coincide.
\begin{center}
    $A \subseteq B \equiv \forall x.(x \in A \rightarrow x \in B)$\\    
\end{center}

\noindent \textbf{Proper (Strict) Subset} \\
\indent A subset is a proper subset if the subset does not contain all of the elements in its superset.
\begin{equation*}
    A \subset B \equiv \forall x.(x \in A \rightarrow x \in B) \land \exists x.(x \in B \land x \notin A)   
\end{equation*}

\noindent \textbf{Set Equality}
\indent Two sets are equal if and only if they have the same elements. Equivalently, if the two sets are subsets of each other, then they are equal.
\begin{equation*}
    A = B \equiv (A \subseteq B) \land (B \subseteq A) \equiv \forall x.(x \in A \iff x \in B)
\end{equation*}

\noindent \textbf{Power Set}

$\mathcal{P}(S)$ is the set of all subsets of S. For example, the powerset of the set $S = \{a,b,c\}$ is:
\begin{equation*}
    \mathcal{P}(S) = \{\emptyset, \{a\}, \{b\}, \{c\}, \{a,b\}, \{a,c\}, \{b,c\}, \{a,b,c\}\}
\end{equation*}
\noindent If $|S| = n$, then $|\mathcal{P}(S)| = 2^{n}$. \\

\noindent \textbf{Cartesian Product} \\
\indent A Cartesian product is a mathematical operation that returns a set of tuples from multiple sets. That is, for sets A and B, the Cartesian product A $\times$ B is the set of all ordered pairs (a, b) where $a \in A$ and $b \in B$. Products can be specified as following:
\begin{equation*}
    A\times B = \{\,(a,b)\mid a\in A \ \land \ b\in B\,\}
\end{equation*}

\noindent In general, $A \times B \neq B \times A$. This is because the tuples are ordered, unlike sets. For example, take the set $A = \{a,b,c\}$ and $B = \{1,2,3\}$.

\begin{gather*}
    A \times B = \{(a,1),(a,2),(a,3),(b,1),(b,2),(b,3)\} \\
    B \times A = \{(1,a),(2,a),(3,a),(1,b),(2,b),(3,b)\}
\end{gather*}

\noindent Also, $|A \times B| = |A||B|$. \\

\noindent \textbf{Union} \\
\indent $A \cup B = \{x\ |\ (x \in A) \lor (x \in B)\}$ \\

\noindent \textbf{Intersection} \\
\indent $A \cap B = \{x\ |\ (x \in A) \land (x \in B)\}$ \\

\noindent \textbf{Difference} \\
\indent $A \setminus B = \{x\ |\ (x \in A) \land (x \notin B)\}$ \\

\noindent \textbf{Complement} \\
\indent $A^{C} = \overline{A} = \{x\ |\ x \notin A\}$ \\

\subsection{Algebra of Sets}
\begin{itemize}
    \item The Idempotent Laws 
    \begin{gather*}
        X \cup X = X \quad \text{and} \quad X \cap X = X
    \end{gather*}
    \item The Commutative Laws
    \begin{gather*}
        X \cup Y = Y \cup X \quad \text{and} \quad X \cap Y = Y \cap X
    \end{gather*}
    \item The Associative Laws
    \begin{gather*}
        X \cup (Y \cup Z) = (X \cup Y) \cup Z \quad \text{and} \quad X \cap (Y \cap Z) = (X \cap Y) \cap Z
    \end{gather*}
    \item The Distributive Laws
    \begin{gather*}
        X \cup (Y \cap Z) = (X \cup Y) \cap (X \cup Z) \quad \text{and} \quad X \cap (Y \cup Z) = (X \cap Y) \cup (X \cap Z)
    \end{gather*}
    \item{The Identity Laws}
    \begin{gather*}
        X \cup \emptyset = X \quad X \cap \Omega = X \quad X \cap \emptyset = \emptyset \quad X \cup \Omega = \Omega
    \end{gather*}
    \item{The Complement Laws}
    \begin{gather*}
        X \cup X^{c} = \Omega \quad X \cap X^{c} = \emptyset \quad (X^{c})^{c} = X \quad \Omega^{c} = \emptyset \quad \emptyset^{c} = \Omega
    \end{gather*}
    \item{The De Morgan's Laws}
    \begin{itemize}
        \item $X\ \backslash \ (Y \cup Z) = (X\ \backslash \ Y) \cap (X\ \backslash\  Z)$
        \item $X\ \backslash \ (Y \cap Z) = (X\ \backslash\ Y) \cup (X\ \backslash\ Z)$
        \item $(X \cup Y)^{c} = X^{c} \cap Y^{c}$
        \item $(X \cap Y)^{c} = X^{c} \cup Y^{c}$
        \item $(\cup_{k=1}^{\infty}X_{k})^{c} = \cap_{k=1}^{\infty}X_{k}^{c}$
        \item $(\cap_{k=1}^{\infty}X_{k})^{c} = \cup_{k=1}^{\infty}X_{k}^{c}$
    \end{itemize}
    
\end{itemize}

\noindent \textbf{Russell's Paradox} \\
\indent Russell's paradox revealed a great but subtle flaw in naive set theory. Consider Russell's set, a set of all sets which don't contain themselves:
\begin{equation*}
    R = \{S\ |\ S \notin S\}
\end{equation*}
\noindent Then if $R \in R$ this yeilds a contradiction, because any element of R cannot contain itself. But if $R \notin R$, this also yields a contradiction because if $R$ doesn't contain itself, it should be an element of $R$ by the definition of $R$.

\newpage
\begin{comment}
======================================================================================================================================================================================================================================================================================REAL ANALYSIS================================================ ====================================================================================================================================================================================================================================
\end{comment}

\section{Real Analysis}

\subsection{Ordering Relations}
\noindent \textbf{Binary Relation} \\
\indent A binary relation between set $X$ and set $Y$ is a subset of $X \times Y$. When $X = Y$ we say the binary relation is on X.
\\
\indent If $x \in X$ and $y \in Y$ are related by binary relation between $X$ and $Y$, we write $xRy$. If $x$ and $y$ are not related by relation $R$, we write $x\cancel{R}y$. \\
\indent A function $f: X \rightarrow Y$ is a binary relation such that if $xRy$ then $x\cancel{R}\Tilde{y}, \forall \Tilde{y}( \neq y) \in Y$.
\\

\noindent \textbf{Ordering Relation} An ordering on a set $X$ is a binary relation, denoted by $\leq$, that satisfies the following properties:

\begin{itemize}
    \item \textbf{Reflexivity}: $x \leq x \ \forall x \in X$.
    \item \textbf{Anti-symmetry}: $x \leq y$ and $y \leq x \implies x = y$.
    \item \textbf{Transitive}: $x \leq y$ and $y \leq z \implies x \leq z$.
\end{itemize}

\noindent $\left \{ X, \leq \right \}$ is called an ordering space and $X$ an order set (sometimes called a partially ordered set. \\

\noindent \textbf{Totally Ordered Set}\\
\indent A totally ordered set is a set in which any two elements can be compared, which leads naturally to the concepts of a maximal and minimal elements. \\


\noindent \textbf{Minimal and Maximal Element}\\
\indent Minimal element: $\underline{m} \in X\ s.t.\ \underline{m} \leq x, \forall x \in X$.\\
\indent Maximal element: $\overline{m} \in X s.t.\ x \leq \overline{m}, \forall x \in X$.
\\

\noindent \textbf{Infimum and Suprimum}\\
\indent Let $(X, \leq)$ be an ordering space, and $\Tilde{X}$ be a non-empty subset of $X$. \\
\indent The infimum is the greatest lower bound of a set $\Tilde{X}$. More formally, a lower bound is a number $\underline{x} \in X$ such that $\underline{x} \leq x\  \forall x \in \Tilde{X}$. If $\underline{x} \leq \underline{x_{0}}$ for any lower bound $\underline{x}$, we call $\underline{x_{0}}$ the infimum. \\
\indent The supremum is the smallest upper bound of a set $\Tilde{X}$. Formally, an upper bound is a number $\overline{x} \in X$ such that $x \leq \overline{x}\ \forall x \in \Tilde{X}$. If $\overline{x_{0}} \leq \overline{x}$ for any upper bound $\overline{x}$, we call $\overline{x_0}$ the supremum.


\subsection{Equivalence Relations}
An equivalence relation on $X$, denoted by $~$, is a binary relation on $X$ such that

\begin{itemize}
    \item \textbf{Reflexivity}: $x \sim x\ \forall x \in X$.
    \item \textbf{Symmetry}: $x \sim y \Leftrightarrow y \sim x$.
    \item \textbf{Transitivity}: $x \sim y$ and $y \sim z \implies x \sim z$
\end{itemize}

\noindent The equivalence class of an element $x$ with the relation $\sim$, denoted by $[x]_{\sim}$, is the set $[x]_{\sim} = \{y|y \in X\ \text{and}\ y \sim x\}$. The quotient set of $X$ with respect to equivalence relation $\sim$ is the set of all equivalence classes,

\begin{gather*}
    X/ \sim = \{[y]_{\sim}|y \in X\}.
\end{gather*}

\noindent For example, let $X = \{x,y,z\}$. We can then define the equivalence relation $~$ on $X$ as

\begin{gather*}
    ~= \{(x,x),(y,y),(z,z),(x,y),(y,x)\}.
\end{gather*}

\noindent Then $[x]_{\sim} = [y]_{\sim} = \{x,y\}$ is an equivalence class and $[z]_{\sim} = \{z\}$ is another equivalence class.
\\

\noindent \textbf{Properties of Equivalence Classes}
\begin{itemize}
    \item (i) $\forall x \in X, x \in [x]_{\sim}$.
    \item (ii) If $y \in [x]_{\sim} \implies x \in [y]_{\sim}$ and $[x]_{\sim} = [y]_{\sim}$.
    \item (iii) $\forall x, y \in X,$ either $[x]_{\sim} = [y]_{\sim}$ or $[x]_{\sim} \cap [y]_{\sim} = \emptyset$.\\
\end{itemize}

\noindent \textbf{Partitions}\\
\indent Let $X$ be a set and $A$ an index set. Let $\{X_{\alpha}\}_{\alpha \in A}$ be a family of nonempty (i.e. $X_{\alpha} \neq \emptyset\ \forall \alpha \in A$) disjoint (i.e. $X_{\alpha} \cap Y_{\beta} = \emptyset\ \forall \alpha \neq \beta \in A$) subsets of $X$. This is called a partition of $X$ if $X = \cup_{\alpha \in A}X_{\alpha}$.\\
\indent Every equivalence relation $\sim$ introduces a partition of $X$, and every partition of $X$ introduces an equivalence relation on $X$.

\subsection{Functions}
\noindent \textbf{Injectivity, Surjectivity, Bijectivity} \\
\indent A function $f(x) : X \rightarrow Y$ can be:
\begin{itemize}
    \item Injective (One-to-one): $x_{1} \neq x_{2} \in X \implies f(x_{1}) \neq f(x_{2}) \in Y$.
    \item Surjective (Onto): $Range(f) = Y$ or $\forall y \in Y,\ \exists x \in X\ s.t.\ f(x) = y$.
    \item Bijective: $f(x)$ is both injective and surjective.\\
\end{itemize}

\noindent \textbf{Characteristic Function} \\
\indent The characteristic function of a set $X$ is defined as
\begin{gather*}
    \chi_{X}(x) =  \begin{cases} 
                      1, & x \in X \\
                      0, & x \in X^{c}
                   \end{cases}
\end{gather*}

\noindent \textbf{Extension and Restriction} \\
\indent Let $f : E \rightarrow F$ and $g : X \rightarrow Y$. If $E \subset X$ and $g(x) = f(x),\ \forall x \in E$, we call $g$ the extension of $f$ to $X$ and $f$ the restriction of $g$ to $E$, denoted by $f = g|_{E}$. \\

\noindent \textbf{Product, Summation, and Composition} \\
\indent Let $f : X \rightarrow Y,\ g : X \rightarrow Y,$ and $h : Y \rightarrow Z$.
\begin{itemize}
    \item Product: $fg(x) = f(x)g(x)$. The domain is $Dom(fg) = Dom(f) \cap Dom(g)$.
    \item Summation: $(f + g)(x) = f(x) + g(x)$. The domain is $Dom(f+g) = Dom(f) \cap Dom(g)$.
    \item Composition: $f \circ g(x) = g(f(x))$. The domain is $Dom(f \circ g) = \{x|x\in Dom(f), f(x) \in Dom(g)\}$.\\
\end{itemize}

\noindent \textbf{Inverse Function} \\
\indent Let $f:X \rightarrow Y$ be a function. We say $f$ is invertible if there exists a function, denoted by $f^{-1}:Ran(f) \rightarrow X$ with the property:

\begin{gather*}
    f(x) = y \iff f^{-1}(y) = x,\ \forall y \in Ran(f)
\end{gather*}

\noindent or equivalently,

\begin{gather*}
    f \circ f^{-1}(y) = y,\ \forall y \in Ran(f)\ \text{and}\ f^{-1} \circ f(x) = x,\ \forall x \in Dom(f).
\end{gather*}

\indent In general, $f^{-1}(x) \neq (f(x))^{-1} = \frac{1}{f(x)}$. The notation $f^{1}$ can also denote the pre-image of a set. To avoid confusion, I will use square brackets to denote pre-image, and round brackets to denote the inverse function. For exmaple, the pre-image of the set $X$ is $f^{-1}[X]$, while the inverse function of $f$ will be $f^{-1}(x)$ at $x$. \\

\noindent \textbf{Invertibility of Injective Functions} \\
\indent If $f:X \rightarrow Y$ be a function that is injective, then $f$ is also invertible. \textit{Proof}:\\
$f$ is injective, so $\forall y \in Ran(f), \exists!x \in Dom(f)\ s.t.\ f(x) = y$. Define $f^{-1}(y) = x$ with $x$ s.t. $f(x) = y$. Then $f^{-1}: Ran(f) \rightarrow X$ is a function and satisfies the requirements of an inverse function for $f$. \\

\noindent \textbf{Uniqueness of Inverse Functions} \\
\indent If $f: X \rightarrow Y$ is a function that is invertible, then its inverse function $f^{-1}$ is unique.

\subsection{Basic Topology}
Topology is concerned with the properites of space that are preserved under continuous deformations, such as stretching, crumpling and bending, but not tearing or gluing. This can be studied by considering a collection of subsets, called open sets, that satisfy certain properties, turning the given set into what is known as a topological space. \\

\noindent \textbf{Cardinality} \\
\indent The cardinality of a set is the number of distinct elements in a set. The cardinality of the empty set is define to be 0 $(|\emptyset| = 0)$. If we let $X$ and $Y$ be nonempty sets, then the cardinality of the sets, $|X|$ and $|Y|$, are numbers that satisfy

\begin{gather*}
    |X| \leq |Y|, |X| = |Y|, |X| \geq |Y|
\end{gather*}

\noindent if there exists $f: X \rightarrow Y$, with $Dom(f) = X$ that is injective, bijective, or surjective respectively. We replace $\leq$ with $<$ when the function is injective but not bijective from $X$ to $Y$, and we replace $\geq$ with $>$ when the function is surjective but not bijective from $X$ to $Y$. \\

\noindent \textbf{Finite, Countable, and Uncountable Sets} \\
\indent For any positive integer $n$, let $J_{n}$ be the set whose elements are the integers $1,2,...,n$. Let $J$ be the set consisting of all positive integers. If there exists a 1-1 mapping of $A$ onto $B$, we say that $A$ and $B$ can be put in a 1-1 correspondence, or that $A$ and $B$ have the same cardinal number, or that $A$ and $B$ are equivalent, and we write $A \sim B$. Then for any set $A$ we say
\begin{itemize}
    \item $A$ is \textbf{finite} if $A \sim J_{n}$ for some $n$.
    \item $A$ is \textbf{infinite} if $A$ is not finite.
    \item $A$ is \textbf{countable} if $A \sim J$.
    \item $A$ is \textbf{uncountable} if $A$ is neither finite nor countable.
    \item $A$ is \textbf{at most countable} if $A$ is finite or countable.
\end{itemize}


\noindent \textbf{Theorems} \\
\indent Infinite subsets of countable sets are countable. \\
\indent Cartesian products of countable sets are countable. If $S$ and $T$ are both countable, then $S \times T$ is countable.\\
\indent Cantor's Theorem - There is no surjective function from $X$ and $\mathcal{P}(X)$ with domain $X$, i.e. $|X| < |\mathcal{P}(X)|$.  \\
\indent Take $X = \mathbb{N}$. Cantor's theorem then says that the collection of all subsets of $\mathbb{N}$ is uncountable. 

\subsection{Metric Space}

\noindent \textbf{Metric Space Definition} \\
\indent Let $X$ be a set and $\rho: X \times X \rightarrow [0,\infty)$ a function. $\rho$ is called a metric if
\begin{itemize}
    \item (i) (Non-negativity) $\rho(x,y) \geq 0,\ \forall x,y \in X.\ \rho(x,y) = 0$ iff $x = y$.
    \item (ii) (Symmetry) $\rho(x,y) = \rho(y,x)\ \forall x,y \in X$. 
    \item (iii) (Triangle Inequality) $\rho(x,y) \leq \rho(x,z) + \rho(z,y)\ \forall x,y,z \in X$.
\end{itemize}
We call $(X,\rho)$ a metric space. The value $\rho(x,y)$ is called the distance between element $x$ and element $y$ in metric $\rho$.
\\

\noindent \textbf{Equivalence of Metrics} \\
\indent Two metrics, $\rho$ and $\sigma$, on a set $X$ are said to be equivalent if $\forall x \in X,\ \forall \epsilon > 0,\ \exists \delta(\epsilon, x)$ s.t. $\forall y \in X$,

\begin{gather*}
    \rho(x,y) \leq \delta(\epsilon, x) \implies \sigma(x,y) \leq \epsilon,
\end{gather*}
\noindent and
\begin{gather*}
    \sigma(x,y) \leq \delta(\epsilon, x) \implies \rho(x,y) \leq \epsilon.
\end{gather*}
\indent Two metrics $\rho$ and $\sigma$, on a set $X$ are said to be uniformly equivalent if there exists positive constants $c_{1}$ and $c_{2}$ such that
\begin{gather*}
    c_{1}\rho(x,y) \leq \sigma(x,y) \leq c_{2}\rho(x,y),\ \forall x,y\in X.
\end{gather*}

\subsubsection{Triangle Inequality}
\begin{gather}
    \lvert x + y \rvert \leq \lvert x \rvert + \lvert y \rvert\\
    |x-y| \geq \big\lvert |x| - |y| \big\rvert
\end{gather}

\subsection{Open and Closed Sets}

\noindent \textbf{Open and Closed Balls} \\
\indent Let $(X,\rho)$ be a metric space. The open and closed balls centered at $x$ with radius $r$ are respectively the sets

\begin{gather*}
    B_{r} (x) \equiv \{y \in X | \rho (x,y) < r\} \quad \text{and} \quad \overline{B}_{r} (x) \equiv \{y \in X | \rho (x,y) \leq r\}.
\end{gather*}

\noindent The open ball $B_{r}$ is usually called the neighborhood of center $x$ with radius $r$.
\\

\noindent \textbf{Open and Closed Sets} \\
\indent Let $(X, \rho)$ be a metric space. A set $S \subseteq X$ is open in $X$ if, $\forall x \in S,\ \exists \epsilon > 0\ s.t.\ B_{\epsilon} \subseteq S$. A set $S$ is closed if its complement $S^{c}$ (relative to $X$) is open. It then follows that the set $B_{r}(x)$ is open and the set $\overline{B}_{r}(x)$ is closed.
\\

\noindent \textbf{Axoim of Open Sets} \\
\indent Let $(X, \rho)$ be a metric space. Then the sets $\emptyset$ and $X$ are both open and closed. Sets that are both open and closed are often called clopen sets.
\\

\noindent \textbf{Union and Intersection of Open Sets} \\
\indent Let $(X, \rho)$ be a metric space and $\{O_{k}\}_{k=1}^{\infty}$ be a countable collection of open sets in $X$. Then

\begin{itemize}
    \item[(i)] $\cap_{k=1}^{N} O_{k}$ is open for any finite $N < \infty$.
    \item[(ii)] $\cup_{k=1}^{\infty} O_{k}$ is open.
\end{itemize}

\noindent \textbf{Union and Intersection of Closed Sets} \\
\indent Let $(X, \rho)$ be a metric space and $\{\Tilde{O}_{k}\}_{k=1}^{\infty}$ be a countable collection of closed sets in $X$. Then

\begin{itemize}
    \item[(i)] $\cup_{k=1}^{N} \Tilde{O}_{k}$ is closed for any finite $N < \infty$.
    \item[(ii)] $\cup_{k=1}^{\infty} \Tilde{O}_{k}$ is closed.
\end{itemize}

\noindent \textbf{Limit Point, Point of Closure, and Isolated Point} \\
\indent Let $(X,\rho)$ be a metric space, and $S \subseteq X$ a subset. A point $x \in X$ is called a limit point of $S$ if $\forall \epsilon > 0,\ \exists y \neq x \in B_{\epsilon} (x)\ s.t.\ y \in S$. We denote $S'$ as the set of limit points of $S$. \par
A point $x \in X$ is called a point of closure of $S$ if $\forall \epsilon > 0,\ \exists y \in B_{\epsilon}(x)\ s.t.\ y \in S$. We denote $\overline{S}$ as the closure of $S$, i.e. the set of points of closure of $S$. A point of closure of $S$ that is not a limit point is called an isolated point. By definition every limit point is a point of closure but not vice versa.

\begin{theorem}
    Let $(X,\rho)$ be a metric space and $S \subseteq X$ a subset. Then $S$ is closed if and only if $S' \subseteq S$.
\end{theorem}

\subsection{Sequences}
\begin{definition}[Sequence]
    Formally, a sequence can be defined as a function whose domain is either the set of the natural numbers (for infinite sequences) or the set of the first $N$ natural numbers (for a sequence of finite length $N$).
\end{definition}
\noindent \textbf{Convergence}\\
\indent Let $(X,\rho)$ be a metric space and $\{x_{n}\}$ a sequence in $X$. We say that $\{x_{n}\}$ converges (in metric $\rho$) to $x \in X$ if 

\begin{gather*}
    \forall \epsilon > 0,\ \exists N(\epsilon) \in \mathbb{N},\ s.t.\ \rho(x_{n},x) \leq \epsilon,\ \forall n \geq N(\epsilon).
\end{gather*}

\noindent When $\{x_{n}\}$ converges to $x$, we write $x_{n} \rightarrow x$ and call $x$ the limit of the sequence in metric $\rho$.
\\

\noindent \textbf{Equivalent Definition of Convergence} \\
\indent Let $(X,\rho)$ be a metric space and $\{x_{n}\}$ a sequence in $X$. Then $x_{n} \rightarrow x$ if and only if $\rho(x_{n},x) \rightarrow 0$ as $n\rightarrow \infty$. 

\begin{theorem}
    Let $\rho$ and $\sigma$ be two equivalent metrix defined on $X$ and $\{x_{n}\}$ be a sequence in $X$. Then $x_{n} \rightarrow x$ in metric $\rho$ iff $x_{n} \rightarrow x$ in $\sigma$.
\end{theorem}

\begin{theorem}[Uniqueness of Limit]
    Let $(X,\rho)$ be a metric space and $\{x_{n}\}$ be a sequence in $X$. Then $x_{n} \rightarrow x$ and $x_{n} \rightarrow y$ then $x = y$.
\end{theorem}

\begin{theorem}
    Let $x_{n} \rightarrow x$ and $y_{n} \rightarrow y$ in metric space $(X,\rho)$. Then $\rho(x_{n},y_{n}) \rightarrow \rho(x,y)$.
\end{theorem}

\begin{theorem}[Sequential Characterization of Limit Point]
    Let $(X, \rho)$ be a metric space and $S \subseteq X$ a subset. A point $x \in X$ is a limit point of $S$ if and only if there exists a non-constant sequence of points $\{x_{n}\} \subseteq S$ s.t. $x_{n} \rightarrow x$ in metric $\rho$.
\end{theorem}

\begin{theorem}[Sequential Characterization of Closedness]
    Let $(X,\rho)$ be a metric space and $S \subseteq X$ a subset. Then $S$ is closed if and only if the limit of every convergent sequence in $S$ belongs to $S$.
\end{theorem}

When the limit of every convergent sequence in $S$ belongs to $S$ it is often said that $S$ is \textit{sequentially closed}. The above theorem says that $S$ is closed if and only if it is sequentially closed.
\\


\subsubsection{Cauchy Sequence} 

\begin{definition}[Cauchy Sequence]
    A sequence $\{x_{n}\}$ in a metric space $(X,\rho)$ is said to be a \textit{Cauchy sequence} if $\forall \epsilon > 0,\ \exists N \in \mathbb{N}$ s.t. $\rho(x_{n},x_{m}) < \epsilon$ for any pair $(n,m)$ s.t. $n \geq N$ and $m \geq N$.
\end{definition}

\begin{theorem}[Convergence Implies Cauchyness]
    Let $(X,\rho)$ be a metric space and $\{x_{n}\}$ be a sequence in $X$. If $x_{n} \rightarrow x \in X$, then $\{x_{n}\}$ is a Cauchy sequence.
\end{theorem}

\begin{proof}
    Using the triangle inequality, we see that $\rho(x_{n},x_{m}) \leq \rho(x_{n},x) + \rho(x,x_{m})$ and the fact that $\rho(x_{n},x) \rightarrow 0$ and $\rho(x,x_{m}) \rightarrow 0$, shows that $\rho(x_{n},x_{m}) \rightarrow 0$.
\end{proof}

\begin{definition}[Completeness]
    A metric space in which every Cauchy sequence converges is said to be \textit{complete}.
\end{definition}

\begin{definition}[Monotonicity]
    A sequence $\{s_{n}\}$ of real numbers is said to be
    \begin{itemize}
        \item[(i)] \textit{monotonically increasing} if $s_{n} \leq s_{n+1}\ (n = 1,2,3,...)$;
        \item[(ii)] \textit{monotonically decreasing} if $s_{n} \geq s_{n+1}\ (n = 1,2,3,...)$;
    \end{itemize}
\end{definition}

\begin{theorem}
    If $\rho$ and $\sigma$ are uniformly equivalent metrics on $X$, then $(X,\rho)$ is complete iff $(X,\sigma)$ is complete.
\end{theorem}

\begin{theorem}[Closed Subspaces Are Complete]
    Let $(X,\rho)$ be a complete metric space and $\Tilde{X}$ a subspace. Then $(\Tilde{X},\rho)$ is complete iff $\Tilde{X}$ is closed.
\end{theorem}
\begin{theorem}[Completeness of Product Spaces]
    Let $(X,\rho_{X})$ and $(Y,\rho_{Y})$ be two complete metric spaces. Define the metric $p: X \times Y \rightarrow [0,\rho)$ as
    \begin{gather*}
        \rho(x,y) = \rho_{X}(x^x,y^x) + \rho_{Y}(x^y,y^y).
    \end{gather*}
    Then $(X\times Y,\rho)$ is a complete metric space.
\end{theorem}

\subsection{Compactness}
There are three different but equivalent ways to study compactness. We will start with the one that is built on the concept of convergence.

\subsubsection{Sequential Compactness}
\begin{definition}(Subsequence)
    Let $(X,\rho)$ be a metric space and $\{x_{n}\}$ a sequence in $X$. Let $n: \mathbb{N} \rightarrow\mathbb{N}$ be an integer function that is strictly increasing (monotonically increasing, i.e. $n(k+1) > n(k),\ \forall k$). We then call $\{x_{n(k)}\}_{k=1}^{\infty}$ (or more simply $\{x_{n_{k}}\}$) a subsequence of $\{x_{n}\}$.
\end{definition}
\subsection{Continuity}
There are four forms of continuity that a function can take. It can be continuous at a point in its domain, continuous on all its domain, uniformly continuous on all its domain, and Lipschitz continuous on all its domain.
\begin{definition}[Continuity]\ \\
    (i) We say that $f$ is continuous at $x_{0} \in X$ if $\forall \epsilon > 0,\ \exists \delta(\epsilon,x_{0}) > 0$ such that
    \begin{gather}
        \rho_{X}(x,x_{0}) \leq \delta \implies \rho_{Y}(f(x),f(x_{0}) \leq \epsilon \quad \forall x \in Dom(f).
    \end{gather}
    If $f$ is continuous at any point $x_{0} \in X$, we say $f$ is continuous.
    (ii) We say that $f$ is uniformly continuous in $X$ if $\forall \epsilon > 0,\ \exists \delta(\epsilon)$ such that
    \begin{gather}
        \rho_{X}(x,x_{0}) \leq \delta \implies \rho_{Y}(f(x),f(x_{0})) \leq \epsilon, \quad \forall x,y\in Dom(f).
    \end{gather}
    (iii) We say that $f$ is Lipschitz continuous in $X$ if $\exists M \in \mathbb{R}$ such that
    \begin{gather}
        \rho_{Y}(f(x),f(y)) \leq M\rho_{X}(x,y), \quad \forall x,y \in Dom(f).
    \end{gather}
    
    The difference between being uniformly continuous, and being simply continuous at every point, is that in uniform continuity the value of $\delta$ depends only on $\epsilon$ and not on the point in the domain.
\end{definition}

\begin{theorem}[Strength of Continuities]
    Let $f:X\rightarrow Y$ be a function. Then
    \begin{itemize}
        \item[(i)] If $f$ is Lipschitz continuous on $Dom(f)$, then $f$ is uniformly continuous on $Dom(f)$.
        \item[(ii)] If $f$ is uniformly continuous on $Dom(f)$, then $f$ is continuous on $Dom(f)$.
    \end{itemize}
\end{theorem}

\subsection{Integral Calculus}
\begin{definition}[Partition on an Interval]
    A partition $P$ of the interval $[a,b]$ is a collection of $N+1\ (N \in \mathbb{N}$ points $\{x_{i}\}_{i=1}^N$ that satisfy:
    \begin{gather}
        a = x_{0} < x_{1} < ... < x_{N-1} < x_{N} = b.
    \end{gather}
    The interval $I_{i} = [x_{i-1},x_{i}],\ (1 \leq i \leq N)$ is called the $i^{\text{th}}$ subinterval in the partition $P$.
\end{definition}

\begin{theorem}
    Let $f$ be bounded on $[a,b]$ and $P$ be a partition with $N$ subintervals. Define
    \begin{gather}
        M_{i} = \sup_{x}\{f(x)|x\in I_{i}\}, \quad \quad m_{i} = \inf_{x}\{f(x)|x\in I_{i}\}.
    \end{gather}
    Define
    \begin{gather}
        U_{P}(f) = \sum_{i=1}^{N}M_{i}(x_{i} - x_{i-1}), \quad \quad L_{P}(f) = \sum_{i=1}^{N}m_{i}(x_{i} - x{i-1}).
    \end{gather}
    Then
    \begin{itemize}
        \item[(i)] $L_{P}(f) \leq U_{P}(f),\ \forall P$.
        \item[(ii)] Let $P$ and $Q$ be two partitions and $P \subseteq Q$. Then $L_{P}(f) \leq L_{Q}(f)$ and $U_{P}(f) \geq U_{Q}(f)$.
        \item[(iii)] $L_{P}(f) \leq U_{Q}(f),\ \forall P,Q$.
        \item[(iv)] $\sup_{P}\{L_{P}\} \leq \inf_{P}\{U_{P}(f)\}$.
    \end{itemize}
\end{theorem}

\begin{theorem}[Reimann Integrable]
    A bounded function $f$ on a finite interval $[a,b]$ is said to be Riemann integrable if
    \begin{gather}
        \sup_{P}\{L_{P}(f)\} = \inf_{P}\{U_{P}(f)\}.
    \end{gather}
    If $f$ is Riemann integrable, we define
    \begin{gather}
        \int_{a}^{b} f(x)dx \equiv \sup_{P}\{L_{P}(f)\} = \inf_{P}\{U_{P}(f)\}.
    \end{gather}
\end{theorem}

\begin{definition}
    If $f$ on a finite closed interval $[a,b]$ is Riemann integrable, then we define
    \begin{gather}
        \int_{b}^{a}f(x)dx = -\int_{a}^{b}f(x)dx.
    \end{gather}
\end{definition}

\begin{theorem}[Riemann Integrability Test]
    Let $f$ be bounded on the interval $[a,b]$. If $\forall \epsilon > 0,\ \exists P$ such that
    \begin{gather}
        U_{P}(f) - L_{P}(f) \leq \epsilon,
    \end{gather}
    then $f$ is Riemann integrable.
\end{theorem}

\begin{theorem}[Continuity on Finite Closed Interval Implies Riemann Integrability]
    Let $f(x)$ be a continuous function on a finite interval $[a,b]$. Then $f$ is Riemann integrable.
\end{theorem}
\begin{proof}
    Let $P$ be a partition with $N$ intervals. Then 
    \begin{gather}
        U_{P}(f) - L_{P}(f) = \sum_{i=1}^{N}(M_{i}-m_{i})(x_{i}-x_{i-1}).
    \end{gather}
    The interval is finite and closed, so by theorem TODO $f$ is uniformly continuous on $[a,b]$. This means $\forall \epsilon/(b-a)>0,\ \exists \delta$ s.t.
    \begin{gather}
        |x-y| \leq \delta \implies |f(x) - f(y)| \leq \frac{\epsilon}{b-a}.
    \end{gather}
    Now pick a partition $P$ such that $\sup_{i}|x_{i}-x_{i-1}| \leq \delta$. This implies that $(M_{i} - m_{i}) \leq \epsilon/(b-a)$. Then $\forall \epsilon>0,\ \exists P$ s.t.
    \begin{gather}
        U_{P}(f)-L_{P}(f) \leq \frac{\epsilon}{b-a}\sum_{i=1}^{N}(x_{i} - x_{i-1}) = \epsilon.
    \end{gather}
\end{proof}

\begin{definition}[Riemann Sum]
    Let $[a,b]$ be a finite closed interval and $f$ a bounded function on the interval. Let $P$ be a partition that has $N$ intervals. Define
    \begin{gather}
        S = \sum_{i=1}^{N}f(x_{i}^{*})(x_{i}-x_{i-1})
    \end{gather}
    with $x_{i}^{*}$ somewhere in $[x_{i-1},x_{i}]$. We call $S$ the Riemann sum for $f$ for partition $P$.
\end{definition}

\begin{theorem}[Rigorous Sequence of Partitions implies Riemann Integral, todo title]
    Let $f$ be a continuous function on a finite interval $[a,b]$ and $\{P_{k}\}$ a sequence of partitions such that $\max_{i}|x_{i}-x_{i-1}| \rightarrow 0$ as $k \rightarrow \infty$. Let $S_{k}$ be any Riemann sum corresponding to $P_{k}$. Then
    \begin{gather}
        S_{k} \rightarrow \int_{a}^{b} f(x)dx \quad \text{as} \quad k \rightarrow \infty.
    \end{gather}
\end{theorem}
\begin{proof}
    $\forall \epsilon/|b-a| > 0,\ \exists \delta > 0$ such that $|x-y| \leq \delta \implies |f(x) - f(y)| \leq \epsilon/|b-a| \implies U_{P}(f) - L_{P}(f) \leq \epsilon$. Since $L_{P_{k}} \leq S_{k} \leq U_{P_{k}}$ and $L_{P_{k}} \leq \int_{a}^{b}f(x)dx \leq U_{P_{k}}$, we can combine these to get
    \begin{gather}
        |S_{k} - \int_{a}^{b}f(x)dx| \leq U_{P_{k}} - L_{P_{k}} \leq \epsilon,\ \forall k \geq K,
    \end{gather}
    with $K$ such that $\max_{i}|x_{i}-x_{i-1}| \leq \delta$. Thus,
    \begin{gather}
        S_{k} \rightarrow \int_{a}^{b}f(x)dx \quad \text{as} \quad k \rightarrow \infty.
    \end{gather}
\end{proof}

\subsubsection{Properties of Riemann Integrals}
Let $f$ and $g$ be continuous functions on the finite interval $[a,b]$. Then we have the following properties:
\begin{itemize}
    \item[(i)] Linearity
        \begin{gather}
            \int_{a}^{b} (\alpha f(x) + \beta g(x))dx = \alpha\int_{a}^{b}f(x)dx + \beta\int_{a}^{b}g(x)dx.
        \end{gather}
        \item[(ii)] Suppose $f(x) \leq g(x) \forall x\in [a,b]$. Then
            \begin{gather}
                \int_{a}^{b} f(x)dx \leq \int_{a}^{b}g(x)dx.
            \end{gather}
        \item[(iii)] Modulus of integral is less than the integral of the modulus.
            \begin{gather}
                \bigg\lvert \int_{a}^{b} f(x)dx \bigg\rvert \leq \int_{a}^{b}|f(x)|dx.
            \end{gather}
        \item[(iv)] Modulus of integral is less than supremum times interval length.
            \begin{gather}
                \bigg\lvert \int_{a}^{b}f(x)dx \leq (b-a) \sup_{x\in [a,b]}|f(x)|.
            \end{gather}
        \item[(v)] Integral can be split onto subintervals.
            \begin{gather}
                \int_{a}^{b}f(x)dx = \int_{a}^{c}f(x)dx + \int_{c}^{b}f(x)dx.
            \end{gather}
\end{itemize}

\newpage
\begin{comment}
======================================================================================================================================================================================================================================================================================COMPLEX ANALYSIS================================================ ====================================================================================================================================================================================================================================
\end{comment}

\section{Complex Analysis}

\subsection{Identities}

\begin{gather*}
    i = \sqrt{-1} \quad \quad \quad z = x + iy \\
    \left | \frac{a}{b}\right | = \frac{|a|}{|b|} \quad \quad \quad |ab| = |a||b|
\end{gather*}

\noindent \textbf{Properties of Complex Conjugate} \\
\begin{align*}
             \overline{z + w} &= \overline{z} + \overline{w} \\
             \overline{z - w} &= \overline{z} - \overline{w} \\
                \overline{zw} &= \overline{z} \; \overline{w} \\
               \overline{z/w} &= \overline{z}/\overline{w},\quad \text{if } w \ne 0 \\
                 \overline{z} &= z ~\Leftrightarrow~ z \in \mathbb{R} \\
               \overline{z^n} &= \left(\overline{z}\right)^n,\quad \forall n \in \mathbb{Z} \\
  \left| \overline{z} \right| &= \left| z \right| \\
         {\left| z \right|}^2 &= z\overline{z} = \overline{z}z \\
      \overline{\overline{z}} &= z \\
                       z^{-1} &= \frac{\overline{z}}{{\left| z \right|}^2},\quad \forall z \neq 0
\end{align*}

\subsection{Triangular Inequality}
\begin{gather*}
    |a + b| \leq |a| + |b|
\end{gather*}

\subsection{Cauchy-Goursat Theorem}

If two different paths connect the same two points in the complex plane, and a function is holomorphic (or analytic) everywhere between the two paths, then the two path integrals of the function will be the same.


\begin{comment}
======================================================================================================================================================================================================================================================================================PROBABILITY THEORY================================================ ====================================================================================================================================================================================================================================
\end{comment}

\newpage

\section{Probability Theory}

\subsection{Multiplication and Addition Principle}
\noindent \textbf{Multiplication Principle} \\
\indent If there are $a$ ways of doing something, and there are $b$ ways of doing another thing, then there are $a\cdot b$ ways of performing both actions. \\

\noindent \textbf{Addition Principle} \\
\indent If we have $a$ ways of doing something and $b$ ways of doing another thing and we can not do both at the same time, then there are $a + b$ ways to choose one of the actions.

\subsection{Combinations vs Permutations}
A combination is how many ways can you pick a number of objects from a group of objects, and permutations are how many ways you can arrange a number of items.

\begin{table}[H]
    \centering
    \begin{tabular}{c | c | c}
        Order Matters & \multicolumn{2}{c}{How many ways to pick r objects from...}\\
        \hline
         & n objects & n types of objects \\
         \hline
        \multirow{2}{*}{Yes} & Permutation & Permutation w/ repetition \\
        & $P(n,r) = \frac{n!}{(n-r)!}$ & $P^{*}(n,r) = n^{r}$ \\
        \hline
        \multirow{2}{*}{No} & Combination & Combination w/ repetition \\
        & $C(n,r) = \frac{n!}{r!(n-r)!}$ & $C^{*}(n,r) = \frac{(n+r-1)!}{r!(n-1)!}$ \\
    \end{tabular}
    \label{tab:my_label}
\end{table}

\noindent For permutations with $n$ types of objects with $k_{1}$ objects of type $1$, $k_{2}$ objects of type $2$, etc, we have
\begin{gather*}
    \frac{k!}{k_{1}!k_{2}!k_{3}!...k_{n}!}
\end{gather*}

\noindent permutations.

\subsection{Binomial and Multinomial Theorem}

\noindent \textbf{Binomial Theorem} \\
\indent The binomial theorem describes the algebraic expansion of powers of any binomial.
\begin{equation*}
    (x+y)^{n} = \sum_{k=0}^{n} {n\choose k} x^{k}y^{n-k}
\end{equation*}
\\

\noindent \textbf{Corollaries from the Binomial Theorem} \\
\begin{gather*}
    \sum_{k=0}^{n} {n\choose k} = 2^{n} \\
    \sum_{k=0}^{n} (-1)^{k} {n\choose k} = 0
\end{gather*}

\noindent \textbf{Pascal's Identity} \\
Best thought of as the rows in Pascal's triangle, the Pascal identity says that
\begin{gather*}
    {n+1 \choose k} = {n \choose k-1} + {n\choose k}.
\end{gather*}

\noindent \textbf{Multinomial Theorem} \\
\indent For any positive integer $m$ and any non-negative integer $n$, the multinomial formula tells us how a sum with $m$ terms expands when raised to an arbitrary power $n$:
\begin{gather*}
    (x_1 + x_2  + \cdots + x_m)^n = \sum_{k_1+k_2+\cdots+k_m=n} {n \choose k_1, k_2, \ldots, k_m} \prod_{t=1}^m x_t^{k_t}\,,
\end{gather*}

\noindent where

\begin{gather*}
    {n \choose k_1, k_2, \ldots, k_m} = \frac{n!}{k_1!\, k_2! \cdots k_m!}
\end{gather*}

\noindent is a multinomial coefficient. For each term in the expansion, the exponents of $x_{i}$ must add up to $n$. 

\subsection{General Rules}

\noindent \textbf{Independence} \\
\indent Let A and B be events with non-zero probabilities. A and B are independent if any (and hence all) of the following hold:
\begin{enumerate}
    \item $P(A|B) = P(A)$, or
    \item $P(B|A) = P(B)$, or
    \item $P(A \cap B) = P(A)P(B)$.
\end{enumerate}

\noindent \textbf{Identities} \\
\indent For two variables,

\begin{gather*}
    P(A \cap B) = P(A) + P(B) - P(A \cup B),
\end{gather*}

\noindent and for three,

\begin{gather*}
    P(A \cup B \cup C) = P(A) + P(B) + P(C) - P(A \cap B) - P(A \cap C) \\ - P(B \cap C) + P(A \cap B \cap C).
\end{gather*}

\noindent \textbf{Conditional Probability} \\
\indent The probability of event A given that event $B$ has already occurred is written as $P(A|B)$.

\begin{gather*}
    P(A|B) = \frac{P(A \cap B)}{P(B)}
\end{gather*}

\noindent If two events A and B are mutually exclusive, then $P(A \cap B) = \emptyset$ and $P(A \cup B) = P(A) + P(B)$.\\

\noindent \textbf{Bayes Theorem} \\
\indent Suppose that the sample space S is partitioned into $n$ disjoint subsets $B_{1},B_{2},...,B_{n}$ with nonempty probabilities, $P_{r}(B_{i}) > 0\ \forall i \in {1,2,...,n}$, and $B_{i} \cap B_{j} = \emptyset\ \forall i \neq j$. Then for an event A,

\begin{gather*}
    P_{r}(B_{j} | A) = \frac{P_{r}(B_{j} \cap A)}{P_{r}(A)} = \frac{P_{r}(B_{j}) \cdot P_{r}(A | B_{j})}{\sum_{i = 1}^{n} P_{r}(B_{i}) \cdot P_{r}(A | B_{i})}
\end{gather*}


\subsection{Discrete Stochastic Variables}

We say $X$ is a discrete stochastic (random) variable if $X$ is a numerically valued function whose domain is the sample space of a probability experiment with a finite or countably infinite number of outcomes. Every random variable has a probability distribution associated with it, and there exists a one-to-one mapping between the two. \\

\noindent \textbf{Median}\\
\indent If $X = \{x_{1},x_{2},...,x_{n}\}$ is a collection of $n$ \textit{sorted} data points then the median of the data is

\begin{gather*}
    M(X) =  \begin{cases} 
                      x_{\cfrac{n+1}{2}}\ , & $n$ \text{ is odd}\\\\
                      \cfrac{x_{\cfrac{n}{2}} + x_{\cfrac{n}{2} + 1}}{2}\ , & $n$ \text{ is even}
                   \end{cases}
\end{gather*}

\noindent \textbf{Midrange} \\
\indent If $X = \{x_{1},x_{2},...,x_{n}\}$ is a collection of $n$ \textit{sorted} data points then the midrange of the data is the minimal element plus the maximal element divided by two:

\begin{gather*}
    \frac{x_{1} + x_{n}}{2}.
\end{gather*}

\noindent \textbf{Mode} \\
\indent If $X = \{x_{1},x_{2},...,x_{n}\}$ is a collection of $n$ \textit{sorted} data points then the mode of the data is the value $x_{i}$ that occurs most frequently. If two values occur the same number of times (and more often than any other value), we say the data is bi-modal. Otherwise, the mode does not exist.
\\

\noindent \textbf{Percentiles} \\
\indent If $X = \{x_{1},x_{2},...,x_{n}\}$ is a collection of $n$ \textit{sorted} data points then $x_{i}$ corresponds to the

\begin{gather*}
    \left ( 100 \cdot \frac{i}{n+1}\right )^{th} \text{ percentile}.
\end{gather*}

\indent Linear interpolation can be used to find the value of a fractional data point.\\

\noindent \textbf{Expectation Value} \\
\indent If $X$ is a discrete random variable with probability function $p(x_{i})$, then the expectation value (mean) of the random variable $X$ is given by,

\begin{gather*}
    E[X] = \sum_{i}^{\infty} x_{i}p(x_{i})
\end{gather*}

\noindent and for an arbitrary function $g(X)$, we call $Y = g(x)$ a transformation of $X$ such that

\begin{gather*}
    E[Y] = E[g(X)] = \sum_{i}^{\infty} g(x_{i})p(x_{i}).
\end{gather*}

\noindent \textbf{Variance} \\
The variance of a discrete random variable is given by
\begin{gather*}
    Var[X] = \sigma_{X}^{2} = \sum_{x_{i}} (x_{i} - \mu_{X})^{2} \cdot p(x_{i}),
\end{gather*}

or

\begin{gather*}
    Var[X] = E[X^2] - (E[X])^{2} = \sum_{x_{1}} x_{i}^{2} p(x_{i}) - \left ( \sum_{x_{i}} x_{i} p(x_{i})\right )^{2} \\
    = \sum_{x_{i}} x_{i}^{2} p(x_{i}) - \mu_{x}^{2}.
\end{gather*}

\noindent \textbf{Mean and Variance of a Linear Transformation} \\
\indent Let $X$ be a discrete random variable and let $Y = a\cdot X + b$, where $a,b \in \mathbb{R}$. Then,
\begin{itemize}
    \item[1.] \quad $E[Y] = E[a\cdot X + b] = a\cdot E[X] + b$.
    \item[2.] \quad $Var[Y] = Var[a\cdot X + b] = a^2 \cdot Var[X]$. 
\end{itemize}

\noindent \textbf{Standard Deviation} \\
Let $X$ be a discrete random variable. The \textit{standard deviation} of $X$ is given by,

\begin{gather*}
    \sigma_{X} = \sqrt{Var[X]} = \sqrt{\sum_{x_{i}} x_{i}^{2} p(x_{i}) - \mu_{X}^{2}} = \sqrt{E[X^2] - (E[X])^2}.
\end{gather*}

\noindent \textbf{Standardized Random Variable} \\
\indent Let $X$ be a discrete random variable and let $Z = \frac{X - \mu}{\sigma}$. Then $Z$ is called the \textbf{standardization of $X$}. The random variable $Z$ always has a mean equal to 0 and a standard deviation equal to 1.
\\

\noindent \textbf{Z-score} \\
\indent Let $x$ be a value from a probability distribution with mean $\mu$ and standard deviation $\sigma$. Then the \textbf{z-score} for $x$ is defined to be:

\begin{gather*}
    z = \frac{x - \mu}{\sigma}.
\end{gather*}

The z-score can be interpreted as a measure for the number of standard deviations away from the mean a data point lies.\\

\noindent \textbf{Markov Inequality} \\
\indent Let $Y$ be a discrete random variable taking only non-negative values, with finite mean $\mu_{Y}$. Then $\forall a > 0$,

\begin{gather*}
    \text{Pr}[Y>a] \leq \frac{\mu_{Y}}{a}.
\end{gather*}

\noindent \textbf{Chebychev's Theorem}\\
Let $X$ be a discrete random variable with finite mean $\mu_{X}$ and standard deviation $\sigma_{X}$. Let $k$ be greater than 1. Then the probability that $X$ is more than $k$ standard deviations from the mean, $\mu_{X}$, is less than or equal to $1/k^2$. That is,

\begin{gather*}
    \text{Pr}(X < \mu_{X} - k \sigma_{X} \ \ \text{or} \ \ X > \mu_{X} + k\sigma_{X}) =  \text{Pr}(\lvert X - \mu \rvert > k\sigma_{X}) \leq \frac{1}{k^2}.
\end{gather*}
\noindent Equivalently,
\begin{gather*}
    \text{Pr}(\mu_{X} - k\sigma_{X} \leq X \leq \mu_{X} + k\sigma_{X}) = \text{Pr}(\lvert X - \mu \rvert \leq k \sigma_{X}) \geq 1 - \frac{1}{k^2}.
\end{gather*}

\noindent \textbf{Outliers} \\
\indent We define an outlier to be any data point with an absolute $z$-score greater than 3 standard deviations away from the mean, i.e. $\lvert z \rvert > 3$.\\

\noindent \textbf{Coefficient of Variation} \\
\indent The coefficient of variation is a statistic that measures the relative variability of a random variable -- the ratio of the standard deviation to the mean. A random variable $X$ with mean $\mu$ and standard deviation $\sigma$ has a coefficient of variation

\begin{gather*}
    \frac{100 \cdot \sigma}{\mu}\%
\end{gather*}

\noindent \textbf{Join Distribution and Independence} \\
\indent Let $X$ and $Y$ be random variables arising from the same discrete probability experiment. The \textbf{joint distribution} of $X$ and $Y$ is given by,

\begin{gather*}
    p(x,y) = \text{Pr}[\{X = x\} \cap \{Y = y\}]
\end{gather*}

\noindent We say $X$ and $Y$ are \textbf{independent} if $\forall x,y$ the events $\{X=x\}$ and $\{Y=y\}$ are independent. That is,
\begin{gather*}
    p(x,y) = \text{Pr}[\{X = x\} \cap \{Y = y\}] = \text{Pr}[X=x]\cdot \text{Pr}[Y=y] = p_{X}(x) \cdot p_{Y}(y)
\end{gather*}

Let $X$ and $Y$ be random variables arising from the same probability experiment. Then,

\begin{gather*}
    E[X + Y] = E[X] + E[Y]. 
\end{gather*}

\noindent This extends to sums of any length. Further, if $X$ and $Y$ are independent, then

\begin{gather*}
    E[X\cdot Y] = E[X] \cdot E[Y], \\
    Var[X + Y] = Var[X] + Var[Y].
\end{gather*}

\noindent This last equation extends to finite sums of any length provided the summands are all pairwise independent.\\

\noindent \textbf{Probability Generating Function} \\
\indent Let $X$ be a random variable taking non-negative integer values. Let $p_{n} = \text{Pr}[X=n]$. Then the probability generating function for $X$ is defined by,
\begin{gather*}
    h(s) = \sum_{i=0}^{\infty} p_{i}s^{i} = E[s^{X}].
\end{gather*}
From this function we can derive the expectation value and the variance of the distribution to be
\begin{gather*}
    E[X] = h'(1), \\
    Var[X] = h''(1) + h'(1) - (h'(1))^{2}.
\end{gather*}

\subsection{Discrete Distributions}
\noindent \textbf{Finite Summation Formulas} \\
\begin{gather*}
    \sum_{i=1}^{n} i = \frac{n(n+1)}{2} \\
    \sum_{i=1}^{n} i^2 = \frac{n(n+1)(2n+1)}{6} \\
    \sum_{n=0}^{N} ax^{n} = \frac{a(1-x^{N+1})}{1-x}
\end{gather*}

\noindent \textbf{Discrete Uniform Distribution} \\
\indent A random variable $X$ is said to have a discrete uniform distribution if its probability function is
\begin{gather*}
    \text{Pr}(X = x) = p(x) = \frac{1}{n} \quad \text{for}\ x = 1,2,...,n
\end{gather*}

\indent The mean and variance are then
\begin{gather*}
    E[X] = \frac{n+1}{2} \\
    Var[X] = \frac{n^2 - 1}{12}
\end{gather*}

\noindent \textbf{Bernoulli Random Variable}\\
\indent A Bernoulli trial is an experiment that has two outcomes, 1 or 0. Thus Pr$[X=1] = p$ and Pr$[X = 0] = q = 1 - p$. The mean and variance are given by
\begin{gather*}
    E[X] = p\\
    Var[X] = pq = p(1-p)
\end{gather*}

\noindent \textbf{Binomial Distribution} \\
\indent Let $Y$ be the number of successes in $n$ independent repetitions of a Bernoulli trial with probability of success $p$. The random variable $Y$ has the probability function given by
\begin{gather*}
    \text{Pr}(Y=y) = p(y) = {n \choose y}p^{y}q^{n-y} \quad \text{for}\ y=0,1,2,...,n\ \text{and}\ 0 \leq p \leq 1.
\end{gather*}

\noindent With probability of success $p$ and $n$ trails is then the mean and variance are:
\begin{gather*}
    \mu_{Y} = E[Y] = np,\\
    \sigma_{Y}^{2} = Var[Y] = np(1-p) = npq.
\end{gather*}

\noindent \textbf{Geometric Distribution} \\
\indent Consider a series of independent Bernoulli trials with probability of success $p$ and let the random variable $X$ be the number of failures before the first success. The random variable $X$ has the probability function given by
\begin{gather*}
    \text{Pr}[X=k] = pq^k
\end{gather*}

\noindent The mean and variance are given by
\begin{gather*}
    E[X] = \frac{q}{p}, \\
    Var[X] = \frac{q}{p^2}.
\end{gather*}

\noindent \textbf{Negative Binomial Distribution} \\
\indent The negative binomial distribution is a generalization of the geometric distribution. The requirements for a negative binomial process are:
\begin{itemize}
    \item[(i)] The trials are identical.
    \item[(ii)] Each trial is independent of any other trial.
    \item[(iii)] The random variable $M$ denotes the number of failures prior to the $r^{th}$ success.
    \item[(iv)]  The probability of success is $p$ and the probability of failure is $q$.
\end{itemize}

\indent The probability distribution is then
\begin{gather*}
    \text{Pr}(M=k) = {r+k-1 \choose k} p^{r}q^{k}.
\end{gather*}

\noindent The mean and variance of this distribution is then,
\begin{gather*}
    E[M] = \frac{rq}{p},\\
    Var[M] = \frac{rq}{p^{2}}.
\end{gather*}

\noindent \textbf{Hyper-Geometric Distribution} \\
\indent A finite population consists of $B$ objects of type 1 and $G$ objects of type 2. Let $X$ be the number of type 1 objects in a sample of size $n$ selected without replacement. Then,
\begin{gather*}
    \text{Pr}(X = k) = \frac{_{B}C_{k} \cdot _{G}C_{n-k}}{_{B+G}C_{n}}
\end{gather*}

\noindent with $0 \leq k \leq B$ and $0 \leq n - k \leq G$. Then $X$ (the number of successes) is called a hyper-geometric random variable. The mean and variance of this distribution is then
\begin{gather*}
    E[X] = n\left ( \frac{B}{B+G}\right ),\\
    Var[X] = n\left ( \frac{B}{B+G}\right ) \left ( \frac{G}{B+G}\right ) \left ( \frac{B+G-n}{B+G-1}\right ).
\end{gather*}

\noindent \textbf{Poisson Distribution} \\
\indent The Poisson distribution with the random variable $Z$ has the probability function given by
\begin{gather*}
    \text{Pr}(Z=k) = e^{-\lambda} \frac{\lambda^{k}}{k!} \quad \text{for}\ \lambda > 0.
\end{gather*}
\noindent with mean and variance given by
\begin{gather*}
    E[Z] = Var[Z] = \lambda
\end{gather*}

Moreover, the sum of two independent Poisson random variables is another Poisson random variable. Suppose that $Z_{i}$ are independent Poisson random variables with mean $\lambda_{i}$ for $i = 1,2$. Then $Z = Z_{1} + Z_{2}$ is a Poisson random variable with mean $E[Z] = E[Z_{1} + Z_{2}] = \lambda_{1} + \lambda_{2}$.\par
The mode of a Poisson distribution is 
\begin{itemize}
    \item[(i)] If $\lambda \in \mathbb{N}$, then the mode is bimodal with $\lambda$ and $\lambda - 1$.
    \item[(ii)] Otherwise, the mode is $\floor*{\lambda}$.
\end{itemize}

\subsection{CDFs and PDFs}
\subsubsection{Cumulative Distribution Function}
The cumulative distribution function (CDF) of a random variable $X$ is defined by,
\begin{gather*}
    F_{X}(x) = Pr(-\infty,x] = Pr[-\infty < X \leq x]
\end{gather*}
The properties of all cumulative distribution functions are
\begin{itemize}
    \item[(i)] $F_{X}(x)$ is a non-decreasing function.
    \item[(ii)] $\lim\limits_{x\to-\infty} F_{X}(x) = 0$.
    \item[(iii)] $\lim\limits_{x\to\infty} F_{X}(x) = 1$.
    \item[(iv)] Pr$(a < X \leq b) = F_{X}(b) - F_{X}(a)$.
\end{itemize}

\subsubsection{Density Function}
If $X$ is a random variable with a differentiable cumulative distribution function $F_{X}(x)$, then the probability density function for $X$ is given by:
\begin{gather*}
    f_{X}(x) = F_{X}'(x).
\end{gather*}
The CDF can be calculated from the probability density function:
\begin{gather*}
    F_{X}(x) = \int_{-\infty}^{x}f_{X}(t)dt.
\end{gather*}
Some properties of the probability density function are,
\begin{itemize}
    \item[(i)] $f_{X}(x) \geq 0$.
    \item[(ii)] $\int_{-\infty}^{\infty} f_{X}(x)dx = 1$.
    \item[(iii)] Pr$[a \leq X \leq b] = \int_{a}^{b}f_{X}(x)dx$.
\end{itemize}

To calculate the expected value of a random variable using the CDF, let $X$ be a non-negative random variable living on the interval $(A,B)$. Then
\begin{gather}
    E[X] = A + \int_{A}^{B}(1 - F(x))dx.
\end{gather}

\subsection{Mixed Distributions}
If a probability experiment involves random variables that are partly continuous and partly discrete, then we call the distribution a \textbf{mixed distribution}. For example, consider a spinning wheel in which is painted 25\text{\%} green which returns the value 1, and the remaining sector is labeled continuously from 0 to 1. Then the outcome $X$ is a continuous random variable $75\%$ of the time, and discrete $25\%$ of the time. \par
Let $Y$ and $Z$ be any two given random variables with CDFs $F_{Y}$ and $F_{Z}$, respectively, and let $p$ be a given number such that  $0 < p < 1$. For any real number $x$ define,
\begin{gather}
    F_{X}(x) = pF_{Y}(x) + (1-p)F_{Z}(x).
\end{gather}
The resulting random variable $X$ is called a \textbf{two-point mixture} of $Y$ and $Z$ with mixing weights $p$ and $1-p$.\par
For a two-point mixture $X$ of $Y$ and $Z$ with mixing weights $p$ and $1-p$, then
\begin{gather}
    E[X^n] = pE[Y^n] + (1-p)E[Z^n].
\end{gather}

\subsubsection{Deductibles and Caps}
A \textbf{deductible} is a minimum amount that a customer must exceed in order to receive reimbursement from their insurance company. For example, if one had a dental insurance plan with a deductible of $\$200$ and one's dental bill costed $\$350$, then one would pay $\$150$ for the bill, and the insurance company would cover the remaining $\$200$. We denote \textit{actual loss amount random variable} $X$ to be how much the customer has to cover, and the \textit{payment random variable} $Y$ reflect how much the insurance company covers. For a deductible of value $d$, $X$ and $Y$ have the following relationship
\begin{gather}
    Y = \begin{cases}
            0 & \text{if}\ \ A \leq X < d\\
            X - d & \text{if}\ \  d \leq X < B.
        \end{cases}
\end{gather}
Given the probability density $f_{X}(x)$ for how much the customer pays, the expected value that an insurance company has to be pay is
\begin{gather}
    E[Y] = \int_{A}^{d}0\cdot f_{X}(x)dx + \int_{d}^{B}(x-d)f_{X}(x)dx = \int_{d}^{B}(x-d)f_{X}(x)dx.
\end{gather}
A \textbf{cap} is a minimum value to which the insurance company will cover. The customer is required to pay the difference for any amount that exceeds the cap. The relationship between $X$ and $Y$ is then
\begin{gather}
    Y = \begin{cases}
            X & \text{if}\ \ A \leq X < C\\
            C & \text{if}\ \ C \leq X < B.
        \end{cases}
\end{gather}
The expected value the insurance company has to cover in this scenario is then
\begin{gather}
    E[Y] = \int_{A}^{C}xf_{X}(x)dx + C\int_{C}^{B}f_{X}(x)dx.
\end{gather}

\noindent \textbf{The CDF Method for Deductibles and Caps}\\
\indent Let $X$ be a continuous loss random variable with domain $(A,B); 0 \leq A < B$, and let $C$ be any number such taht $A < C < B$.
\begin{itemize}
    \item[(i)] Benefit capped at $c$: \\
        \begin{gather}
        Y^{c} = \begin{cases}
                    X & \text{if}\ \ A \leq X < c\\
                    c & \text{if}\ \ c \leq X < B.
                \end{cases}
        \end{gather}
    \item[(ii)] Benefit with deductible of $d$\\
        \begin{gather}
            Y_{d} = \begin{cases}
                        0 & \text{if}\ \ A \leq X < d\\
                        X - d & \text{if} \ \ d \leq X < B\\
                    \end{cases}
        \end{gather}
    \item[(iii)] Benefit capped at $c$ with deductible $d$\\
        \begin{gather}
            Y_{d}^{c} = \begin{cases}
                            0 & \text{if}\ \ A \leq X < d\\
                            X - d & \text{if}\ \ d \leq X < c\\
                            c - d & \text{if}\ \ c \leq X < B.
                        \end{cases}
        \end{gather}
\end{itemize}
\begin{itemize}
    \item[(i)] $X = Y^{c} + Y_{d}$.
    \item[(ii)] $E[Y^{c}] = A \int_{A}^{c}(1 - F_{X}(x))dx$.
    \item[(iii)] $E[Y_{d}] = \int_{d}^{B}(1 - F_{X}(x))dx$.
    \item[(iv)] $Y_{d}^{c} = Y^{u} - Y^{d}$.
    \item[(v)] $E[Y_{d}^{c}] = \int_{d}^{c}(1 - F_{X}(x))dx$.
\end{itemize}

\subsubsection{Moment Generating Function}
Let $X$ be any random variable. The \textbf{moment generating function} (MGF) of $X$ is denoted by $M_{X}(t)$ and is defined by,
\begin{gather}
    M_{X}(t) = E_{X}[e^{tX}].
\end{gather}

Some properties of the moment generating function are:
\begin{itemize}
    \item[(i)] \textbf{Moments}: $M_{X}^{(k)}(0) = E[X^k]$.
    \item[(ii)] \textbf{Linear Transformation}: If $Y = aX + b$, then $M_{Y}(t) = e^{bt}M_{X}(at)$.
    \item[(iii)] \textbf{Sums of Independent Random Variables}: If $X_{1},...,X_{n}$ are independent random variables and $S = X-{1} + ...+ X_{n}$, then
        \begin{gather}
            M_{S}(t) = M_{X_{1}}(t) \cdot M_{X_{2}}(t)\cdot\cdot\cdot M_{X{n}}(t).
        \end{gather}
    \item[(iv)] \textbf{Corollary to (iii)}: If $X_{1},...,X_{n}$ are independent random variables, all with common distribution $X$, then $M_{S}(t) = [M_{X}(t)]^n$.
\end{itemize}

Let $X$ be a random variable with $MGF$ $M_{X}(t)$. Define $h(t) = \ln(M_{X}(t))$, then
\begin{gather}
    E[X] = h'(0) \quad \quad \text{and} \quad \quad Var[X] = h''(0).
\end{gather}

\subsubsection{MGF for Some Discrete Distributions}
\noindent \textbf{MGF of a Binomial}\\
Let $S$ be a binomial random variable with $n$ trials and probability of success $p$. Then
\begin{gather}
    M_{S}(t) = (q + pe^t)^n.
\end{gather}

\noindent \textbf{MGF of a Geometric Random Variable}\\
Let $X$ be the number of failures before the first success in a sequence of independent identical Bernoulli trials with probability of success $p$. Then $X$ is a geometric distribution with MGF giben by,
\begin{gather}
    M_{X}(t) = \frac{p}{1 - qe^t}.
\end{gather}

\noindent \textbf{MGF for the Ngeatvie Binomial Variable}\\
Let $S$ be the number of failures before the $r^{th}$ success in a sequence of independent identical Bernoulli trials with probability of success $p$. Then $S$ is a negative binomial distribution with MGF given by,
\begin{gather}
    M_{S}(t) = \left( \frac{p}{1-qe^t} \right)^r.
\end{gather}

\noindent \textbf{MGF for the Poisson Distribution}\\
If $X$ is Poisson with parameter $\lambda$, then the MGF for $X$ is given by,
\begin{gather}
    M_{X}(t) = e^{\lambda(e^t - 1)}.
\end{gather}

\subsection{Continuous Distributions}
\subsubsection{Uniform Distribution}
The \textbf{standard uniform} distribution, denoted by $U$, is given by a constant probability density function on the interval $0$ to $1$.
\begin{gather}
    f(u) = \begin{cases}
                1 & 0 \leq u \leq 1\\
                0 & \text{otherwise}.
            \end{cases}
\end{gather}
The CDF is given by
\begin{gather}
    F(u) = \begin{cases}
                0 & -\infty < u < 0\\
                u & 0 \leq u \leq 1\\
                1 & 1 < u < \infty.
           \end{cases}
\end{gather}
Given the standard uniform distribution, we can define the \textbf{general uniform} distribution as follows.
\begin{gather}
    X = A + (B-A)U.
\end{gather}
The PDF is
\begin{gather}
    f_{X}(x) = \frac{1}{B-A} \quad \quad A \leq x \leq B.
\end{gather}
The CDF is
\begin{gather} 
    F_{X}(x) = \frac{x-A}{B-A} \quad \quad A \leq x \leq B.
\end{gather}
The mean and variance are given by
\begin{gather}
    E[X] = \frac{A+B}{2} \quad \quad Var[X] = \frac{(B-A)^2}{12}.
\end{gather}
If $A \leq a \leq b \leq B$, then 
\begin{gather}
    P[a \leq X \leq b] = \frac{b-a}{B-A}.
\end{gather}

\subsection{Moments}
The $n$th moment of $X$ for a discrete variable is defined as:

\begin{gather*}
    \braket{x^n} = \sum_{i=1}^{j}x^{n}P(x_{i})
\end{gather*}

\noindent where the first moment is just the average over the sample space, and the second moment minus the first moment squared is the variance of $X$. More generally, we can write

\begin{gather*}
    \braket{f(x)} = \sum_{i=0}^{\infty} f(x) P(x_{i})
\end{gather*}

\indent For a continuous variable, the sums turn into integrals to cover the entire sample space. The $n$th moment of $X$ is then defined as

\begin{gather*}
    \braket{x^n} = \int_{-\infty}^{\infty} x^n P(x) dx
\end{gather*}

\noindent and more generally we can write

\begin{gather*}
    \braket{f(x)} = \int_{-\infty}^{\infty} f(x) P(x) dx
\end{gather*}

\indent The standard deviation is then equal to

\begin{gather*}
    \sigma = \sqrt{\braket{(\Delta x)^{2}}} = \sqrt{\braket{x^2} - \braket{x}^2}. 
\end{gather*}

\subsection{Standard deviation}

The standard deviation can be calculated as follows.
\begin{gather*}
    (\Delta j)^2 = (j - \braket{j})^2 \\
    \braket{(\Delta j)^{2}} = \sum (\Delta j)^2 P(j) = \sum (j-\braket{j})^2 P(j) \\
    = \sum (j^2 - 2j\braket{j} + \braket{j}^2) P(j) = \sum j^2 P(j) - 2\braket{j}\sum j P(j) + \braket{j}^2 \sum P(j) \\
    = \braket{j^2} - 2\braket{j}^2 + \braket{j}^2 = \braket{j^2} - \braket{j}^2 = \sigma^{2}
\end{gather*}

\end{document}
